\documentclass[12pt]{article}% or something else
\usepackage{pdfpages}
\usepackage{enumitem}

\usepackage{soul}

\usepackage[a4paper, total={6in, 8in}]{geometry}	

\newcommand{\red}[1]{\textcolor{red}{#1}}

\date{}
%\date{Exam Date: Thu, Jun 27 at 17:00\\
%Location:	ASC 140}

\usepackage[T1]{fontenc}
\usepackage{titling}
\setlength{\droptitle}{-6em}   % This is your set screw


\begin{document}


\title{Review for DATA/COSC 301/DATA 501 Exam}
\maketitle
%\textbf{\Large Review for DATA/COSC 301/DATA 501 Exam}\\ 

\vspace{-3cm}


\subsection*{Proposed Format}

Time limit: 180 minutes\\
Total marks: 100 

\begin{itemize}[topsep=0pt, noitemsep]
\item $\sim$  30 marks for 30 multiple choice questions (1 mark each)%(1 mark each) (30 minutes total)
\item $\sim$ 20 marks for 7 short answer questions
\item $\sim$ 50 marks for 7 long answer questions
\end{itemize}


\section*{Previous Exam Breakdown}

Last year, the exam has covered 36\% midterm 1 material, 27\%  midterm 2 material, leaving 37\% of the material unique to final. % This was the breakdown from last year.  Clearly ours will not follow the exact structure as we have not covered the same amount of material.  
\noindent \textcolor{red}{I am not bound to the exact percentages, but they should give you a rough idea on how to study.}
There is some variation in this distribution each semester, but there will be roughly a 1:1:1 coverage of Midterm 1, Midterm 2 and post Midterm 2 material. 


\begin{table}[h]
\caption{\footnotesize MC = Multiple Choice (1 mark each), SA = Short Answer, LA = Long Answer}
\begin{center}
\begin{tabular}{|c | l| l |}
 \hline
{\bf Percentage}&  {\bf Topic} &  {\bf Distribution} \\
 \hline
   2\% &  Data representation &  2 MC\\
11\% & Excel &  1 MC, 1 LA (10)\\
 1\% & Excel VBA &  1 MC\\
16\% & Databases &  2 MC, 1 SA (4), 1 LA (10)\\
 6\% & Command line$^1$ &  1 MC, 1 LA (5)\\
27\% & Python & 12 MC, 3 LA (3x5)\\
21\% & R &  7 MC, 1 SA, 1 LA (10)\\
 7\% & GIS\footnote{} &  1 MC, 2 SA \\
 5\% & Visualization &  3 MC, 1 SA \\
 4\% & Open data &  2 SA \\
 \hline
\end{tabular}

{\small $^1$ Since we did not cover Command Line and GIS, these percentages will be redistributed across the remaining topics}
\end{center}
\label{default}
\end{table}%

%\section*{Formula Sheet}
%
%In the past I have {\it not} granted a formula sheet for this exam. I will put it to a vote if we want to include on this year. Please keep in mind the following before you vote:
%\begin{itemize}
%\item {\bf Con:} A formula sheet will imply harder marking (eg less forgiving on syntax) and harder questions (eg. little to no definition-type questions).
%\item {\bf Pro} Creating a formula sheet (1 page hand written front and back) provides a security blanket and valuable studying technique.
%\end{itemize}


\newpage


\section{List of Topics}


\begin{table}[h]
\caption{Relevance of topics as it pertains to the final exam}
\begin{center}
\begin{tabular}{|c|l|}
\hline
{\tt ***} & Extremely important\\
 {\tt **} & Assignment question or major topic\\
  {\tt *} &Important topic which probably should be tested\\
    -& (no stars) topic covered but of minor importance\\
   \st{strikethrough} & Will not be tested on the exam\\ 
\hline
\end{tabular}
\end{center}
\label{default}
\end{table}%







\subsection*{Introduction}

\begin{itemize}[noitemsep]
 \setlength{\itemsep}{1pt}%
    \setlength{\parskip}{1pt}
\item[*]  what is data analysis? what does a data analyst do?
\item[*]   importance of data analytics
\end{itemize}


\subsection*{Data Representation}

\begin{itemize}[noitemsep]
 \setlength{\itemsep}{1pt}%
    \setlength{\parskip}{1pt}
%    \item[-]  using the correct terminology
\item[*] Define: computer, software, memory, data, memory size/data size, cloud
\item[*] Explain ``Big Data" and describe data growth in the coming years
\item[**] How data is measured: bits vs bytes vs KB, GB, \dots
\item[*] Compare and contrast: digital versus analog
\item[**] How integers, doubles, and strings are encoded
\item[**] Difference between unsigned binary, float (32 bit), double (64 bit)
\item[**] Convert integer into unsigned binary
\item[-] Convert real number into float
\item[*] Why ASCII table is required for character encoding
\item[*] Explain why Unicode is used in certain situations instead of ASCII
\item[**] Explain the role of metadata for interpreting data
\item[*] Define: file, file encoding, text file, binary file
    \item[-]  Encode using  Hexadecimal, the NATO broadcast alphabet
    \item[-]  Discuss the time-versus-space tradeoff
    \end{itemize}




\subsection*{Excel}
\begin{itemize}
 \setlength{\itemsep}{1pt}%
    \setlength{\parskip}{1pt}
\item[*] Explain what a spreadsheet is and its usefulness
\item[**] Excel notation: ribbon, worksheet, workbook
\item[**]  spreadsheet cell addressing (eg. range notation using :)
\item[-]  selecting cells in a spreadsheet with your mouse 
\item[-]  filling, hiding
\item[**] Define and explain: formula, function, argument, concatenation
\item[**] Using functions, eg. concatenate, lookup, index, if, aggregate functions
\item[***] compare absolute vs. relative addresses ; use absolute/relative addresses
\item[**] use conditional formatting, format painter
\item[**] data and type formats
\item[**] Use sorting and filtering (with numbers or characters)
\item[**] Create/edit/identify different charts
%\item[-]  using chart features: trendlines, sparklines
 \item[*]  Create and edit charts and use chart features: trendlines, sparklines
\item[*] Explain the usefulness of: what-if scenarios, goal seek, solver
\item[***] Use and create pivot tables and charts (understand the ROW, COLUMN, FILTER, and VALUES panels)
\item[**] Evaluate and create conditions. 
\item[**] Use IF() to make decisions
\item[*] Excel add-ins: define, how to install (eg. Solver. Data Analysis)
\item[-] Linear regression
\end{itemize}

\subsection*{Excel VBA}
\begin{itemize}
 \setlength{\itemsep}{1pt}%
    \setlength{\parskip}{1pt}
\item[***] Explain/understand how to create and use macros and macro recorder
\item[**] Excel file extensions: mainly know the difference between .xlsx and .xlsm
\item[*] Saving macros: Personal workbooks (.xlsb) vs regular .xlsm
\item[**] Explain the security issues with macros and how to handle them
\item[**] Visual Basic Editor: basic definition and features 
\begin{itemize}
\item[*] How to use the immediate window, eg \verb|?/PRINT/DEBUG.PRINT <command>| 
\item[-] VBA modules
\item[-] Object browser
\end{itemize}
\item[*] Be able to read/manipulate macro/VBA code (don't need to create from scratch but you should know the general syntax)
\begin{itemize}
\item[-]  Object-oriented definitions (object, class, property, method) %and objects in Excel
\item[*] The hierarchy of objects eg. Application $\rightarrow$ Workbook $\rightarrow$ Worksheet
%\item[**] Create and use Excel variables
\item[*] Collections and indexing, eg. {\tt Worksheets("macro")}, {\tt Worksheets(2)} (index starting at 1)
\item[**] Create and use Excel variables
\item[*]  Explain how a collection is different from a typical variable
\item[*] eg. {\tt Range} for selecting cells
\end{itemize}
%\item[-] Use If/Then/Else syntax to make decisions (this concept will be tested in R/Python)
%\item[-] Use For loop for repetition (this concept will be tested in R/Python)
%\item[**] \st{Create} user-defined functions and use them in formulas
 \item[**] Explain how to create user-defined functions and use them in formulas
\item[*] Subroutines vs user-defined functions.
%\item[*] Define: event
%\item[-]  List some typical user interface controls
%\item[-]  Understand that Excel allows for forms and controls to be added to a worksheet which respond to events
% \item[**] Difference between subroutine and functions
%\item[*]  Define: event
%\item[-]List some typical user interface controls
%\item[-] Understand that Excel allows for forms and controls to be added to a worksheet which respond to events
\end{itemize}

\subsection*{Relational Databases}
\begin{itemize}
 \setlength{\itemsep}{1pt}%
    \setlength{\parskip}{1pt}
\item[**] Define: database ({\it data}), database system ({\it software}), schema, metadata
\item[***] Define: relation (table), attribute (field/column name), tuple (a single record/row), domain (data type for a column), degree (number of columns/attributes), cardinality (number of rows/tuples)
\item[*] SQL properties: reserved words, case-insensitive, free-format
%\item[-] Using LibreOffice Base/ Microsoft Access specific GUI/shortcuts
\item[-] GUI commands in Microsoft Access and LibreOfficeBase (i.e Design View)
\item[***] Write queries using \underline{SQL} commands%, and update and create tables
\begin{itemize}
\item[N.B] Some commands differed slightly between LibreOffice Base and Access; both will be accepted.
\end{itemize}
\item[***] Be able to create a table using CREATE TABLE 
\begin{itemize}
\item[***] Know your field types (eg. \verb|VARCHAR| vs \verb|CHAR|)
%\item[*] Explain what a key is and what it is used for
\item[**] Explain what a primary key is and what it is used for and how to assign one
\end{itemize}
\item[*] Use DROP TABLE to delete a table and its data
 \item[**] Use INSERT/UPDATE/DELETE to add/update/delete rows of a table %and perform same actions using Microsoft Access/Libre Office Base user interface.
\item[*] ALTER command for adding new columns
\item[***] Projection operation using SELECT
\begin{itemize}
\item[**] DISTINCT to remove duplicates
\end{itemize}
\item[***] Selection operation using WHERE
\begin{itemize}
\item[***] comparison operators (\verb|<, >, =, !=, <=, >=|), 
\item[***] logical operators (\verb|AND, OR, NOT|).
\item[-] \verb|IS NULL|
\end{itemize}
\item[***] JOINS (inner, left, right, outter -- know the difference)
\item[**] Sort rows using ORDER BY (ASC/DESC for ascending/descending, resp.)
\item[**] Use LIMIT/TOP to keep only the first (top) N rows
\item[***] Use GROUP BY and aggregation functions (eg. \verb|COUNT|, \verb|MAX|)  for calculating summary data
\item[**] HAVING for filtering after GROUPBY.
\item[***]  Statement Written/Execution Order 
\end{itemize}


\subsection*{07Python}

\begin{itemize}
 \setlength{\itemsep}{1pt}%
    \setlength{\parskip}{1pt}
\item[-] Explain what is Python and note the difference between Python 2 and 3 (I will be marking using Python 3 syntax)
\item[*] Define: algorithm, program, language, %programming
\item[**] Follow Python basic syntax rules including indentation/comments/colons
\item[-] Jupyter Notebook
\item[***] Using \verb|print| statements
\begin{itemize}
\item[-] Use new-style string formatting, (feel free to use, but you don't need to) \newline
eg. \verb|print("Total score for {} is {}".format(name, score))|
\end{itemize}
\item[*] Perform math expressions and understand operator precedence
\item[***] Print formatting 
\item[***] Use strings, character indexing, string functions, slicing
\item[**] Commonly used string functions/methods: split, substr, concatenation (\verb|+|), escape character
\item[***] Define and use variables and assignment
\item[**] Python naming rules
\item[*]  Python data types (see using {\tt type} function)
\item[***] Python indexing \verb|[]| \textcolor{red}{starts at 0}, -1 right to left indexing
\item[**] Using functions and methods, eg. \verb|len()|, \verb|.append()|, \verb|str()|, \verb|int()|, \verb|.split()|
\item[*] Deleting objects {\tt del}, {\tt .remove}
\item[**] How to import and use modules
\item[-] Use Python datetime and clock functions
\item[***] Read input from standard input (keyboard)
\item[***] Create comparisons and use them for decisions with {\tt if}
\item[**] Combine conditions with {\tt and}, {\tt or}, {\tt not}
\item[**] Order of operations for logical operators, eg {\tt and} before {\tt or}
\item[***] {\tt if/elif/else} statements
\item[***] Looping with {\tt  for} and {\tt while} (eg. using {\tt for} loops with Python lists, strings., etc.)
\begin{itemize}
\item[***] {\tt range(start (inclusive), end (exclusive), step (optional)}
\end{itemize}
\item[***] Create and use lists, list functions (eg. {\tt .sort}), and list slicing, traversing in {\tt for} loop
\item[*] Python collections: tuples, sets
\item[*] Differences between lists, tuples, sets and dictionaries
\item[***] Create and use dictionaries
\item[-]  Advanced: list comprehensions
\item[***] Create and use Python functions
\item[-] Difference between Python functions and procedures
\item[**] Differences between functions and methods
\item[-] lambda functions
\item[-] Use built-in functions in math library
\item[-] Create random numbers
\item[-] Advanced: passing functions, lambda functions
\end{itemize}



\subsection*{08Python}

\begin{itemize}
 \setlength{\itemsep}{1pt}%
    \setlength{\parskip}{1pt}
\item[***] Open, read, write, append to files
\item[**] Closing files (either in a {\tt with} clause or using {\tt fileobj.close()})
\item[***] Process CSV files: you may either use base Python \textit{or} use the csv module (but this module makes it easier)
\item[**] Define: exception, exception handling
\item[***] Use try-except statement to handle exceptions and understand how each of try, except, else, finally blocks are used
\item[-] Define: IPv4/IPv6 address, domain, domain name, URL
\item[-] Read URLs using urllib.request.
\item[-] \st{Use Biopython module to retrieve NCBI data and perform BLAST}
\item[*] Build charts using matplotlib (eg alter existing code to produce some desired affect)
\item[-] Perform linear regression and k-means clustering using SciPy
\item[-] Connect to and query the MySQL database using Python
\item[-] NumPy arrays
\item[*] Write simple Map-Reduce programs
\end{itemize}


\subsection*{Open Data}
\begin{itemize}
 \setlength{\itemsep}{1pt}%
    \setlength{\parskip}{1pt}
%\item[*] List some of the governments and organizations that provide data in an open fashion.
%\item[**] Use open data sets when applicable when performing data analysis.
%\item[*] Explain the purpose and usefulness of Google Analytics and Google Ads
\item[***] Key concepts/take home message from Dr. Jason Pither's guest lecture
\begin{itemize}
\item[***] Define open data and explain the motivations for making data ``open''.
\item[*] Terminology (eg. reproducibility)
\item[*] Publication bias
\item[-] Power/ any technical details
\item[*] Ideal workflow
\item[**] pre-registration of research plan
\item[*] Tools for helping with Open Science
\end{itemize}


\end{itemize}

\subsection*{R}
\begin{itemize}
 \setlength{\itemsep}{1pt}%
    \setlength{\parskip}{1pt}
\item[**] Understand purpose and usefulness of R
\item[**] Types of data: qualitative, quantitative
\item[***] Describe data using numerical summaries (measure of centre/spread, fivenum)
\item[***] Define and calculate: mean, median, variance, standard deviation, range
\item[**] Define: quantile, quartile, interquartile range, five number summary
\item[**] Explain what a  working directory is and how to set it %get it
\item[***] {\sf R} syntax, for example: \verb|{}| for blocking lines of code, case sensitive, \verb|<-|/\verb|=| for assignment character,  \dots
\item[*] Perform matrix addition, subtraction, and multiplication
\item[-] Install and use RStudio (no RStudio specific questions)
\item[**] Write small programs/commands in {\sf R} that may use variables, conditions, logicals, loops, and functions
\item[**] Read in data sets from files (eg. using {\tt read.csv} or {\tt read.table})
\item[**] Use {\tt head} and {\tt tail}% to explore a data set
\item[**] Create and use data structures: vectors, matrices, lists
\item[**] Indexing vectors/matrices/arrays/lists, \textcolor{red}{\it remember that indexing starts from 1 in {\sf R}!}
\item[**] Use data frames/factors for data analysis
\item[*] Create graphs/visualizations: frequency table, bar chart, histogram, boxplot, %ECDF 
using ggplot2 or base (eg. alter code, map code to figure, expected output)
\item[**] List the assumptions of a $t$-test
\item[**] Compute and read output of linear models using R
\item[-] Using SQLish functions (eg. {\tt subset}) may be helpful when writing code, but you %don't {\it need} to use/know them
won't be tested on the explicitly 
\item[*] Handling missing data
\item[-] {\tt repeat}, {\tt next}, {\tt break}
\item[-] {\tt apply()} functions
\item[-] $k$-means
\item[***] Perform hypothesis testing and read output 
\begin{itemize}
\item[***] Stating the null/alternative hypothesis
\item[*] identifying  test statistic 
\item[***] Decision and conclusion based on $p$-values
\item[***] Choosing the most appropriate test (one-sample, independent two-group, paired two-group)
\item[-] Explain the purpose of confidence intervals
\item[***] linear regression
\begin{itemize}
\item[***] State the fitted regression line
\item[*] residuals
\item[*] State the {\sf R}-squared values 
\item[***] Predict $y$ for a given $x$
\end{itemize}
\end{itemize}

\end{itemize}

\subsection*{\st{GIS}}
\begin{itemize}
 \setlength{\itemsep}{1pt}%
    \setlength{\parskip}{1pt}
\item[*] \st{Provide examples where a GIS is used}\\
%\item[**] Define GIS and list some of its features/components
%\item[*] Appreciate history of GIS including Canadian connection
%\item[**] List and use GIS features: text, point, line, polygon
%\item[*] Explain the relationship between features, coordinates, and attributes
%\item[*] Provide an example on how interval and categorical data is displayed
%\item[***] Define: scale, precision, resolution and perform simple calculations
%\item[***] Define: feature class, layer
%\item[**] Compare and contrast raster versus vector representations
%\item[**] Define and use latitude and longitude
%\item[*] Explain the challenge in modeling a point on the earth's surface given that it is not a perfect sphere and has topography
%\item[*] Explain role and connection between a geoid, spheroid, datum
%\item[*] Explain the purpose of a projection and understand different projections have different benefits and distortions
%\item[*] Apply a map design process to produce visually appealing maps
%\item[*] Define and use KML
%\item[**] Create a map with Google My Maps with various features
\vdots
\item[-] \st{Use Python to connect to Google Maps API}
\end{itemize}



\subsection*{\st{Command Line}}
\begin{itemize}
 \setlength{\itemsep}{1pt}%
    \setlength{\parskip}{1pt}
\item[**] \st{Define command line and list some of its uses}\\
%\item[*] Explain the purpose of an operating system
%\item[**] Know how to open the command line window on Mac OS and Windows
%\item[**] Be able to enter commands and stop them
%\item[*] Define: file system, folder, file
%\item[**] Explain the difference between an absolute and relative path
%\item[*] Use command line shortcuts to save time
%\item[*] Be able to match wildcards involving ? and *
%\item[*] Be able to cancel a command
%\item[**] Explain standard input, standard output, and standard error
%\item[*] Be able to use input and output redirection and pipes (\verb|?, >, < , >>|)
%\item[*] Explain the reason for an escape symbol
%\item[*] Define and explain the purpose of environment variables
%\item[*] Be able to use grep to search text files
%\item[*] Explain the purpose of a batch program
\vdots
\item[**] \st{Be able to connect to another machine using SSH}
\end{itemize}



\subsection*{Visualization}
\begin{itemize}
 \setlength{\itemsep}{1pt}%
    \setlength{\parskip}{1pt}
%\item[**] Explain the purpose of visualization
\item[*] List different types of visualizations available in Excel, Python, {\sf R} \st{GIS}
\item[**] Create basic plots using Excel, Python, and {\sf R}
%\item[**] List the three "types of data"
%\item[**] Define: pill, shelf, view card (as used in Tableau)
%\item[*] Explain the purpose of the Show Me button
%\item[**] Be able to connect to Excel and relational databases using Tableau
%\item[*] Compare/contrast connecting to versus extracting data with Tableau
%\item[*] List and explain the different Tableau file types
%\item[**] Define and compute: inner join, left outer join, right outer join, full outer join
%\item[*] Use dynamic grouping and renaming to clean and correct data values in a visualization
%\item[***] List and use the different Tableau chart types: text tables, maps, heat maps, tree maps, line charts, pie charts, area charts, scatter plot, circle view, histogram, Gantt charts
%\item[*] Add trend lines, references lines, quantiles to a visualization
%\item[*] Create and use hierarchies
%\item[**] Create and use filters
%\item[**] Create calculated fields
%\item[*] Use parameters to allow user-controlled visualizations
%\item[*] Add forecasts to a visualization
%\item[**] Organize visualizations into a dashboard
\end{itemize}





\end{document}